\documentclass[11pt, english]{memoir}

\usepackage{babel}
\usepackage{amssymb}
\usepackage{amsmath}
\usepackage{amsfonts}
\usepackage[utf8]{inputenc}   % LaTeX pour les accents 
\usepackage[T1]{fontenc}      % LaTeX pour les accents 
\usepackage{icomma}
\usepackage{framed}
\usepackage{geometry}
\usepackage{graphicx}

\usepackage{enumitem}
\setlist{nosep} % or \setlist{noitemsep} to leave space around whole list
\renewcommand\labelitemi{--}


%Pour redéfinir le format des titres de sections
%\usepackage{titlesec}
%\titleformat{\subsection}{\normalfont\sffamily\bfseries\LARGE\raggedright}{\thesection}%{1em}{}
\renewcommand{\chaptitlefont}{\normalfont\scshape\bfseries\Huge\raggedright}
%Une autre option est la commande suivante
%\setsecheadstyle{\normalfont\bfseries\scshape\Large}

%pour le niveau de "profondeur" de la table des matières
\usepackage{tocloft}
\setsecnumdepth{subsection}

%pour la couleur des hyperliens de la table des matières 
\usepackage[colorlinks]{hyperref}
\hypersetup{linkcolor=blue}

%Pour, entre autre, pouvoir mettre une bordure sur des équations sur plusieurs lignes 
\usepackage{tcolorbox}
\tcbuselibrary{theorems, breakable}
\tcbset{colback = white, colframe = black, arc = 0mm, breakable}


\newtheorem{definition}{définition}
\numberwithin{definition}{section} 

\newtheorem{theorem}{Théorème}
\newtheorem{conjecture}{Propriété}
\newtheorem{example}{Exemple}[section]
\newtheorem{proposition}{Proposition}
\newtheorem{remark}{Remarque}

\newenvironment{proof}[1][Démonstration]{\noindent\ }{\ \rule{0.5em}{0.5em}}
\newenvironment{exemple}
{
	\begin{example} \normalfont \ \\[5pt] 
	}
	{
		\hfill\rule{0.5em}{0.5em}\end{example}
}

\newenvironment{solution}
{\noindent\textbf{Solution:} \\[5pt] 
}{
}


\geometry{headsep=15pt}
\normalsize\setlength{\parskip}{\baselineskip}
\setlength{\oddsidemargin}{25mm}
\setlength{\evensidemargin}{25mm}
\setlength{\voffset}{-1in}
\setlength{\hoffset}{-1in}
\setlength{\textwidth}{165mm}
\setlength{\topmargin}{0mm}
\setlength{\headheight}{15mm}
\setlength{\headsep}{11mm}
\setlength{\topskip}{0mm}
\setlength{\textheight}{222mm}


% Pour enlever l'INDENTATION 
\setlength\parindent{0pt}



% POUR DÉFINIR UNE NOUVELLE COMMANDE (ici l'opérateur de MOYENNE EMPIRIQUE) 
\newcommand{\mean}[1]{\bar{#1}}

\newcommand{\Var}{\text{Var}}
\newcommand{\Cov}{\text{Cov}}
\newcommand{\E}{\text{\textbf{E}}}


\newenvironment{smalltext}
{\footnotesize }{\normalsize}


\begin{document}
	
\title{Module 1 : Summary}
\date{}
\author{François Déry}
\maketitle

	
	


\tableofcontents{}
\newpage


\chapter{Introduction to Risk Management}
\section{Understanding and Quantifying Risk}


Risk can be used in many contexts in risk management and insurance and can have any of the following meanings:

\begin{enumerate}
	\item The subject matter of an insurance policy, such as a structure, an auto fleet, or the possibility of a liability claim arising from an insured’s activities;
	\item The insurance applicant (the insured);
	\item The possibility of bodily injury or property damage;
	\item A cause of loss (or peril), such as fire, lightning, or explosion;
	\item The variability associated with a future outcome.
\end{enumerate}

Within the definition, there are always to elements : the \emph{uncertainty of outcome} (what/when is the event)  and the \emph{possibility of negative outcome}. 

It's important to note that possibility does not quantify risk : one needs to know the \emph{probability} of the event in order to adequately quantify it. In contrast, possibility only dictates if an outcome may or may not occur. 







\section{Risk Classification}

In order to better understand and manage risks, organization typically categorize risks using one or more of following categories : 
\begin{enumerate}
	\item Pure and speculative risk; 
	\item Subjective and objective risk;
	\item Diversifiable and nondiversifiable risk;
	\item Quadrant of risk (hazard, operational, finanacial and strategic).
\end{enumerate}


\subsection{Pure and Speculative Risk}

A \emph{pure risk} is a chance of a loss or no loss, but \emph{no chance of gain}. For example, the fire of a fire loss : the building can burn of not burn. Either way, the owner of the building won't gain anything from the risk. As they represent no opportunity for financial gain, \emph{pure risks are always undesirable}. 

In contrast, \emph{speculative risk involves a chance of gain}. For example, venture capitalism is a speculative risk : it represents a financial risk which has enormous profit potential. Hence, this type of risk is desirable. Other examples are : 
\begin{itemize}
	\item Price risk;\\
	\begin{smalltext}
		Possible changes in material costs, for example.
	\end{smalltext} 
	\item Credit risk;
	\item Financial investments. 
\end{itemize}

The financial investment are usually categorized into four types of speculative risk : 
\begin{enumerate}
	\item Market risk ; \\
	\begin{smalltext}
		Risk associated with the fluctuation in price of stocks and bonds.
	\end{smalltext} 
	\item Inflation risk ;
	\item Interest rate risk ;
	\item Liquidity risk. 
\end{enumerate}


\subsection{Subjective and Objective Risk}

The differentiation between the two comes from the source of assessments regarding the risks : opinions (which are \emph{subjective}) or facts (\emph{objective}).

The closer an organization's subjective interpretation of risk is to an objective risk, the more effective its risk management will be. 

Here are the reason why subjective and objective risk can differ substantially  : 
\begin{enumerate}
	\item Familiarity and control;
	\item Consequences over likelihood ;\\
	\begin{smalltext}
		Either underrating the probability of a low probablitity event, of overstating the probability. In both cases, the assessment of the consequence over the likelihood is wrong.
	\end{smalltext}
	\item Risk awareness ;\\
	\begin{smalltext}
		Of course, if you're not aware that a risk exists, the subjective risk will be wrong.
	\end{smalltext}
\end{enumerate}













	
\end{document}

















